
%%% Local Variables: 
%%% mode: latex
%%% TeX-master: t
%%% End: 

\chapter{总结与展望}
\label{cha:intro}

\section{总结}
\label{}
本文针对存在环结构特征的图像,提出了一种局部不变特征,即环结构特征,能够满足平移、缩放与旋转不变性,可应用于图像识别、图像配准等各个领域。本文的主要工作有:
\begin{enumerate}
\item 环结构特征提取与描述。阐述了提取及描述环结构的主要步骤,即先用多尺度分割算法与骨架化算法提取图像中的点与线,滤除不能组成环结构的特征点,从而找到特征点之间的连接关系。应用我们提出的基于空间的深度优先算法检测环结构,并用分叉角度与分支长度来把环结构描述成特征向量。

\item 基于环结构特征的视网膜图像配准。阐述了如何把环结构特征应用在图像配准中,即在得到环结构特征向量后,用相似性度量进行特征向量匹配,找到最匹配的环结构特征对,再用相似性变换把一幅图像变换到另一幅图像中。提出了首先使用环上的分叉点及与环相连的血管点来做初始全局配准,再用局部未配准分叉点进行局部纠正的策略。提出骨架化对准精度来自动选择最佳配准结果,也可用来对配准结果进行定量评估。论文中还给出了不同变换模型、不同特征、不同方法之间的实验对比,以此来说明用环结构来做视网膜图像配准的可行性和有效性。

\item 基于环结构特征的扇贝图像识别。构建了扇贝图像识别图像库,包括标准图像库与待识别图像库,并通过相似性度量进行特征匹配,找到标准图像库中与待识别图像库中相对应的扇贝图像,通过实验证明,用环结构特征来进行扇贝识别是可行和有效的。
\end{enumerate}
\section{展望}
\label{}
本文提出了一种新的局部不变特征,即环结构特征,以应用于图像内容中存在环结构特征的各类图像,并以视网膜图像配准及扇贝图像识别为例,较为完善的提出了一整套的配准及识别方案,但仍有一些工作需要做进一步研究:
\begin{enumerate}
\item 关于环结构检测算法,目前实现了对于三点、四点、五点环的检测,检测结果准确且算法较为高效,但针对于图像中更多的点组成环结构的情况,采用动态路径移动算法不能进行很好的检测,因此需要进一步考虑更多的情况以完善算法。
\item 关于图像分割,我们采用多尺度分割的方法能得到成不同尺度的分割结果,不同尺度的结果能从不同角度分析出图像的更多信息以加以应用,但分割结果越多,对图像的后续处理越复杂,增加了很多步骤,增大了算法的复杂性,因此需要在尺度的选择及算法的复杂度之间进行权衡。
\item 关于特征提取,本文针对环特征结构,提出用组成环的分叉点的角度及分支长度来把环特征描述成特征向量。但分叉点的角度个数不同,组成环的分叉点个数不同造成了特征向量的长度不同,因此在进行配准或识别时造成了很大的不便。本文提出以4个角度作为标准,不足四个角度则补充为0的思想虽能实现应用的目的,但不够科学,需开展进一步研究,采用更加合适的策略。
\item 针对扇贝图像识别,扇贝图像来自于网络搜集,并未对真正养殖的扇贝进行图像采集。真正的扇贝表面存在一些附着物且扇贝纹理并不如网络搜集的扇贝图像清晰,今后应该采集一些养殖扇贝的图像以做更进一步的研究。
\end{enumerate}
