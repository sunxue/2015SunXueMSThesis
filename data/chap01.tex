
%%% Local Variables:
%%% mode: latex
%%% TeX-master: t
%%% End:

\chapter{绪论}
\label{cha:intro}


\section{引言}
\label{}

近年来,随着计算机视觉领域的不断发展,特征提取作为最重要的一环也得到了大量学者的重视和研究\cite{shipeng}。在计算机视觉发展初期,人们往往通过直方图等全局特征来获取整幅图像的信息\cite{liuying},然而,全局特征对于有部分遮挡、背景杂乱的图像具有很强的敏感性。于是,这就刺激了研究人员对局部特征的研究。局部特征主要利用图像中显著的局部信息来描述图像,局部特征可能是点、边缘、或者区域,通过特征检测技术和特征描述技术将局部特征描述为特征向量,就可把繁杂的图像匹配的问题转换为特征向量的度量问题,这样就能大大减小算法的复杂度,提高算法的有效性。好的局部特征应该满足几个属性:首先,对于从不同角度,不同时间拍摄的图像,应能保证局部特征的有效和不变性,其次,局部特征与图像背景之间应该有较大差异,能够方便和准确的进行提取,最后,局部特征应具有一定的数量,以便能够在图像中检测足够多的特征进行应用。

局部不变特征的这些属性决定了其在计算机视觉各个领域能得到广泛应用,比如在图像识别、图像配准、目标跟踪、图像检索、图像拼接等领域都需要局部特征的提取、描述与匹配。

本文提出了一种新的局部不变特征,即环结构特征,以应用于图像中存在环结构特征的各类图像,比如建筑物图像、树叶图像、视网膜图像、扇贝贝壳图像等。环结构特征由图像中存在的交叉、分叉点及它们之间的连线构成,用归一化的连线长度及分叉角度等信息将环结构描述成特征向量,可满足平移、缩放、旋转不变性,因此可应用于图像识别、图像配准等各个领域。


\section{国内外研究现状}
\label{} 



\subsection{图像中的局部不变特征}
\label{}

局部特征就是从图像的局部区域出发,用显著的局部信息来构造出具有稳定性的描述子。局部特征描述了图像局部区域的信息,由于图像各个区域之间存在像素、颜色或纹理等方面的差异,局部特征体现出了唯一性。近些年来,计算机视觉领域获得迅猛发展,局部特征技术也引起了足够的重视,很多学者纷纷加入研究,越来越多的局部特征描述子在图像配准、物体识别、图像检索等领域获得了大量的应用。

Harris角点特征检测方法\cite{harris}是经典的局部特征检测算法,Harris算法是由Moravec\cite{moravec}算法发展而来,Morave角点检测公式如式\ref{eq:morave},其中$w(x, y)$是高斯平滑因子,种子像素点$I(x, y)$平移$(u, v)$窗口后的灰度变化为E:

\begin{align}
E(u, v)|_{(x, y)} = \sum_{x, y}w(x, y)[I(x + u, y + v) - I(x, y)]^2 
\label{eq:morave}
\end{align}
上式中。基于Morave算法,Harris提出了新的改进,为了抑制噪声,对图像进行了平滑处理。将式\ref{eq:morave}进行泰勒级数展开,得到:
\begin{align}
E(u, v)|_{(x, y)} \doteq [u, v]M\left[ \begin{array}{l}
u \\
v
\end{array} \right]
\end{align}
\begin{align}
M = \sum_{x, y} w(x, y)\left[ \begin{array}{ll}
I_x^2 & I_xI_y\\
I_xI_y & I_y^2
\end{array} \right]
\end{align}
上式中忽略了泰勒级数的高阶项,$I_x$和$I_y$分别为种子像素点在x和y方向上的导数。将M做相似对角化处理,得到:
\begin{align}
M \to R^{-1}\left[ \begin{array}{ll}
\lambda_1 & \\
 & \lambda_2
\end{array} \right]R
\end{align}
其中,$\lambda_1, \lambda_2$是M的特征值,R为旋转因子。Harris的核心思想是:在水平、竖直两个方向上变化均较大的点为角点,即此时两个特征值都比较大,仅在水平或垂直一个方向有较大的变化量为边缘点,即此时有一个特征值较大,当在水平、竖直两个方向的变化量都较小,即两个特征值都相对较小时为平滑区域。这个核心思想可表示为:
\begin{align}
\left\{ \begin{array}{l}
\textrm{Corness = detM - $k(traceM)^2$} \\
\textrm{detM = $\lambda_1\lambda_2$} \\
\textrm{traceM = $\lambda_1 + \lambda_2$}
\end{array} \right.
\end{align}
其中,k是一个经验值,当检测到Corness大于某一设定值且为局部极值点时,可认为是角点。

文献\cite{yezhiyong}认为Harris角点检测算法需要合适的阈值来检测理想的特征点,而阈值的获取依靠大量的实验才能确定,选取不合适则会造成聚类现象的产生。且文献\cite{zhangyong}认为Harris对噪声敏感,对特征点的定位不够准确。

Lowe提出一种具有尺度不变的局部特征,即SIFT算子\cite{lowe}。该方法将图像与不同高斯尺度的变换核相卷积,得到不同尺度下的图像。该算法即使在图像缩放尺度比较大的情况下,也能稳定可靠的提取特征点,且对光照具有一定的适应性。二维搞死卷积核如下:
\begin{align}
G(x, y, \sigma) = \frac{1}{2\pi\sigma^2}e^{-(x^2+y^2)/2\sigma^2}
\end{align}
其中,$\sigma$表示高斯函数分布的方差,不同尺度的图像可表示为:
\begin{align}
L(x, y, \sigma) = G(x, y, \sigma) \otimes I(x, y)
\end{align}

$L(x, y, \sigma)$表示尺度为$\sigma$的图像,$\sigma$表示尺度因子,用来控制图像平滑的程度,$\sigma$越大表示图像平滑的越大。

除此之外,国内外还提出了很多优秀的局部特征描述子,比如Hessian,Susan\cite{smith},SURF\cite{bay}算法等等,这些算子都是比较通用的算法,可以针对不同种类的图像加以应用,但针对某一类图像,可能效果不是很理想。本文提出了一种新的稳定的局部不变特征,即环结构特征,以应用于图像中存在环结构特征的各类图像,将其描述成特征向量,用于图像识别、图像配准等领域。由于环结构特征更具有针对性,故而应用于图像中存在环结构的各类图像,效果更加理想。

\subsection{图论中的环结构检测}
\label{sec:complicatedtable}

环结构是图论中的概念。在人类社会的实际生活中,有时在描述某些事物或对象之间有某种特定关系时采用图形的方式显得更加直观。对象用图形中的点表示,两对象之间具有的某种特定的关系用两点之间的无向或有向连线表示,由此数学抽象产成了图的概念\cite{xujunming}。环在图论中表示一系列边的集合,起始点与终止点是重合的。

在图中检测最小环是图中的研究的最基本的问题。针对这个问题已有很多方面的应用及研究。比如电路测试、结构工程、计算机程序频率分析、有机化学中的复杂合成等各个方面。国内外学者针对检测最小环的问题提出了很多算法。

Mateti和Deo\cite{mateti},Syslo\cite{syslo}提出了检测图中的简单环的算法,Dixon和Goodman\cite{goodman}提出了寻找最长环的算法。Chua和Chen提出把寻找环的问题应用于电路网络中\cite{chua}。上述算法中,工作量大都取决于所选择的环基。若一个环基中所包含的是最小环,则能加快算法的速度,这就刺激了一些学者开始研究检测最小环的算法。最初,Stepanec和Zykov提出了检测最小环的算法\cite{stepanec},Hubicka和Syslo\cite{hubicka}提出,Stepanec的算法不适用于所有的图,并提出了自己的算法。但Kolasinska\cite{kola}给出了一个Hubicka检测最小环基失败的案例。Steeves提出了一种不同的算法,但仍有失败的案例。Deo,Prabhu和Krishnamoorthy\cite{deo}提出这个问题是个N-P问题。

第一个提出多项式时间算法来解决检测最小环基问题的是Horton\cite{horton},复杂度是$O(m^3n)$,de Pina\cite{pina}提出了一个不同的$O(m^3+mn^2logn)$的算法,Golynski和Horton\cite{golynski}通过利用快速矩阵乘法改进了Horton的算法,复杂度为$O(m_{\omega}n)$,$\omega \le 2.376$。Berger\cite{berger}利用与de Pina相同的思想提出了一个复杂度为$O(m^3+mn^2logn)$的算法。Kavitha\cite{kavitha}等人改进了Pina的算法,复杂度为$O(m^2n+mn^2logn)$。Mehlhorn\cite{mehlhorn}提出了一种复杂度为$O(m^2n^2)$的混合算法,之后又基于最小反馈点集提出$O(m^2n)$的算法,最终又改进到$O(m^2n/logn + mn^2)$。

这些算法都是用于解决图论中的问题,为了在图像中能够检测最小环,我们提出了动态路径移动算法。我们首先检测了图像中的分叉结构作为特征点,然后根据特征点之间的连接关系检测环结构,实验证明利用动态路径移动算法能成功的检测到图像中的环结构。

\section{课题来源}
\label{sec:theorem}

本课题主要受国家自然科学基金项目“基于视觉注意结合生物形态特征的海洋浮游植物显微图像分析(编号:61301240)”与“基于形态特征的中国海常见有害赤潮藻显微图像识别(编号:61271406)”资助。


\section{主要工作及内容安排}
\label{sec:bib}

本文的主要工作如下:

第一章为绪论,主要介绍了局部不变特征的相关知识及国内外研究现状、图论中环结构检测的主要研究现状、课题来源以及本文的主要工作和内容安排等。

第二章介绍了环结构特征的定义、特点及应用,以及如何在图像中提取和描述环结构特征。提出了提取及描述环结构的主要步骤,即先用多尺度分割算法与骨架化算法提取图像中的点与线,滤除不能组成环结构的特征点,从而找到特征点之间的连接关系。应用我们提出的动态路径移动算法检测环结构,并用分叉角度与分支长度来把环结构描述成特征向量。

第三章主要把环结构特征应用于视网膜图像配准中。介绍了近年来视网膜配准的研究现状,并阐述了如何把环结构特征向量应用在图像配准中,即用相似性度量进行特征向量匹配,再用相似性变换把一幅图像变换到另一幅图像中。提出了首先使用环上的分叉点及与环相连的血管点来做初始全局配准,再用局部未配准分叉点进行局部纠正的策略,并给出了具体步骤。同时,还给出了不同变换模型、不同特征、不同方法之间的实验对比,以此来说明用环结构来做视网膜图像配准的可行性和有效性。

第四章主要介绍如何把环结构特征用于扇贝图像识别上。介绍了国内外扇贝图像识别的研究现状,构建了扇贝图像识别图像库,并用环结构进行实验来说明用环结构来进行扇贝图像识别的有效性。

第五章是对本论文工作的分析总结以及对今后所要进行工作的展望。
